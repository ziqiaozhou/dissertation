\abstract{
Noninterference, a strong security property for a computation process,
informally says that the process output is insensitive to the value of
its secret inputs -- the secret inputs do not "interfere" with those
outputs.  This is too strong, however;  a degree of interference is
necessary in almost all real systems.  In this dissertation, we
propose a \textit{measure} of noninterference that is more practical.
Based on a model of computations with three types of input (secret,
attacker-controlled, and others) and an attacker-observable output, we
define a noninterference measure that can assess and explain
information leaks in actual codebases.

We start with assessing a new defense against cache-based side-channel
attacks in a cloud environment, using an experiment-based quantitative
measure of leakage against existing attacks.  It is not enough to
measure leakage through empirical analysis, however, as it fails to
identify new interference introduced by a weak defense design.  We
propose a symbolic execution framework to formally measure
interference in simple software procedures, encompassing any
interference from a set of secret inputs to observable outputs.
Leveraging approximate model counting techniques, we make this
framework scalable with parallelization.  Unfortunately, this
technique does not scale to support analysis of hardware processor
designs, in part due to its reliance on symbolic execution to create a
logical postcondition of the computation.  We thus modified the
framework to sidestep symbolic execution when analyzing processor
designs. To further tame the complexity due to various sources of
interference, we extend our framework to remove, or declassify,
certain interference from consideration, so that the framework instead
highlights other forms of interference, and to provide
human-interpretable rules that explain the conditions under which
interference occurs. We demonstrate the practicality of the work
through case studies of both software-based leakage and
vulnerabilities in the RISC-V  BOOM core with different
configurations. 
\iffalse
Considering hardware
and software logic together reminds us that some interference is
required for proper functioning while multiplying the possible leakage
sources. We extend the definition of noninterference in
hardware/software settings to identify allowed interference and to
generate rule-based interpretations to explain unintended
interference.
\fi
}
