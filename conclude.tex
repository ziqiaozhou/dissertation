\chapter{\uppercase{Conclusion}}
Any computation with insecure information flows is potentially
vulnerable to side-channel attacks.  Noninterference was conceived as
a requirement to eliminate any such flows.  In practice, however,
absolute noninterference can rarely be achieved.  This
dissertation has thus made contributions toward the development of
\textit{measuring} noninterference for real-world programs.

First, we explored noninterference assessment using
empirical evaluations against cache-based side-channel
attacks. Relying on these empirical evaluations, we demonstrated
\cachebar's effectiveness in defending against cache-based side
channels in \glspl{LLC}. However, model checking revealed that
empirical assessment was not enough, as it failed to capture
interference not triggered by the concrete experiments.  This lesson
motivated the use of formal approaches for assessing noninterference
more holistically.
\iffalse
%% This is too detailed at this point in the thesis
Concretely,
when the copy-on-access transitions \accessed to \shared due to
timeout or \exclusive to the \shared due to copy merging, the attacker
still could \Reload to learn whether the victim used that memory page
if the those state transitions does not clear the victim's footprint
in cache.
\fi

Second, we suggested a static method for measuring interference from
actual codebases. One contribution of our measurement is its
formulation of interference as the distinguishability of two sets of
secret values.  This novel metric supports noninterference measurement
in multiple dimensions, reflecting how often secrets leak when the
size \secretsSetSize of secret sets is small and how much is leaked
when \secretsSetSize is large.  Case studies showed that our
measurement framework has moderate runtime costs, which range from
minutes to days depending on the workload and the computation
resources (i.e., parallelization is possible).

Third, we extended the static framework to relatively complicated
computations including the processor on which they execute, and
implemented them in \thirdsysname.  By leveraging the declassification
and interpretation capabilities of \thirdsysname, we measured and
explained hardware-software vulnerabilities. \thirdsysname analyzes a
sequence of instructions running for hundreds of cycles above the
\boom processor.  Logical rules sorted by their precision and recall
values explain the sources of leakage without forcing the analyst to
diagnose the leakage from the measurement value alone.

We demonstrated the possibility of using our frameworks to measure and
interpret unintended leakage in practice using our case studies on
side-channel leakage through shared caches, traffic analysis, adaptive
compression algorithms, shared TCP network counters, sliding-window
modular exponentiation algorithms, and speculative executions.  We
hope that these demonstrations of noninterference measurement will
bring \glsfirst{QIF} closer to practice and help analysts develop better
mitigations.
